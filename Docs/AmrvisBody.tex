\section{Amrvis}

\subsection{Overview}

{\em Amrvis} is a visualization and data analysis tool for
examining data files generated by the Adaptive Mesh Refinement
({\bf AMR}) code.  The user can view color planar images of the data,
{\bf AMR} grid regions, data values in a chosen format, subregions, animations,
volumetric renderings, and output in postscript.  It works for 2D and 3D data
and requires no specialized graphics hardware.

It also has batch capabilities for extracting arbitrary subregions
(including cubes, planes, lines, and points) from data files and
preprocessing volume renderings.  Batch capabilities work in parallel
with threads.

A separate tool, {\em AmrMovie}, can be used to view, animate, and output volume
renderings.  This tool can animate preprocessed (with {\em Amrvis}) volume
data at arbitrary orientations and through time.


\subsection{Using Amrvis}

\subsubsection{Command Line Interface and Utility Files}

From the command line, the user can start {\em amrvis} with or without
data (or plot) files and can perform batch operations
to extract data.  The simplest usage is to type {\em amrvis} at the command
line with no arguments.  This will start the main window from which
the user can open data files or quit the application.  Using the {\bf -help}
argument will cause {\em amrvis} to print a usage message and exit.  The usage
is currently

\begin{verbatim}
amrvis [-help] [-b <small x> <small y> <big x> <big y>]
       [-maxpixmapsize <max picture size in # of pixels>]
       [<file type flag>] [-a] [-v] [-bw n] 
       [-xslice n] [-yslice n] [-zslice n] [-sliceallvars]
       [-sgi | -alpha | -reverse2]
       [-palette palname] [-initialderived dername]
       [-initialscale n] [-showboxes tf] [-numberformat fmt]
       [<filename(s)>]
  -b                 specify subdomain box (on finest level).
  -maxpixmapsize     specify maximum allowed picture size in pixels.
  file type flags:   -cgmf, -cgsf, -mf, -sf, -old2, -iamr,
                     -cgiamr, -boxdata, -boxchar, -fab
  -a                 to load files as an animation.
  -bw n              specify maximum boundary width.
  -skippltlines n    skip n lines at head of the plt file.
  -boxcolor n        set volumetric box color value.
  -nobody            do not show the CartGrid body.
  -xslice n          write a fab slice at x = n (n at the finest level).
  -yslice n          write a fab slice at y = n (n at the finest level).
  -zslice n          write a fab slice at z = n (n at the finest level).
  -sliceallvars      write all fab variables instead of just initialderived.
                     Note:  slices are written in batch mode.
  -sgi               set fabordering = FAB_NORMAL_ORDER.
  -alpha             set fabordering = FAB_REVERSE_ORDER.
  -reverse2          set fabordering = FAB_REVERSE_ORDER_2.
  -v                 verbose.
  -palette palname   set the initial palette.
  -initialderived dername   set the initial derived to dername.
  -initialscale n    set the initial scale to n.
  -showboxes tf      show boxes (the value of tf is true or false).
  -numberformat fmt  set the initial format to fmt (ex:  %4.2f).
  <filename(s)>      must be included if box is specified.
\end{verbatim}


File type flags are as defined as follows.  Once a file type is specified,
it cannot be changed for that {\em amrvis} process.

\begin{verbatim}
-cgmf               Cartesian grid multi-fluid plot file (default).
-cgsf               Cartesian grid single-fluid plot file.
-mf                 Multi-fluid plotfile.
-sf                 Single-fluid plotfile.
-old2               Very old amr plot file (type 2).
-iamr               Incompressible plot file.
-cgiamr             Cartesian grid incompressible plot file.
-fab                An FArrayBox.
\end{verbatim}

{\em Amrvis} looks for a default definition file named  {\bf amrvis.defaults}
and if it exists, uses its options as defaults.
If the file does not exist, {\em amrvis} uses standard defaults.
See the section {\bf Files and Directories} for a complete description
of the default options.
The search path for {\bf amrvis.defaults} is (in this order):

\begin{verbatim}
  ./amrvis.defaults
  ~/amrvis.defaults
  ~/.amrvis.defaults
\end{verbatim}



\subsubsection{Interface Windows and Regions}

\begin{description}
\item[Main Window]  This is the small window used to open data files
or quit the main application.  It also prints messages on which file
is being read and values for clicked data points.  There is only one
of these windows per {\em amrvis} application.

\item[Data View Window]  This is the window used to view and interact
with a data set or subregion of a dataset.  It contains the color image,
the palette, and data manipulation controls.

\item[Dataset Window]  This window shows data values and is invoked
with the Dataset button on the currently selected subregion.

\item[Control Area]  This is the area at the top of the Data View Window
which contains controls for selecting current scale, current quantity,
datasets, subregions, and output.

\item[Data Image Area]  This area shows the color image of the dataset.
In 3D, this area is split into four sections--the three orthogonal
planes and an isometric view.

\item[Palette Area]  This area shows the palette which maps data values
to colors and shows the high and low data values that are currently mapped.

\item[Animation Control Area]  This area contains controls for animating
through data files and through planes in 3D.  It is in the lower right
part of the Data View Window below the Palette Area.  It also shows
plane axes.  The animation controls will only be visible in 3D or
if the Data View Window is used for 2D animation.
\end{description}


\subsubsection{Interface Controls}

\begin{description}
\item[Scale Menu]  This menu controls the current image scale.  At
scale 1, a datapoint at the finest displayed level is mapped to a pixel.

\item[Level Menu] This menu allows viewing data on coarser levels.

\item[Quantity Menu]  This menu allows the user to select the current
quantity.  This menu is built from information contained in the plot
file.

\item[Dataset Button]  This button creates a Dataset Window for the
currently selected region.

\item[Subregion Button]  This button creates a new Data View Window for the
data contained in the currently selected region.

\item[Palette Button]  This button allows the user to select a new palette.

\item[Boxes Button]  This button toggles showing {\bf AMR} boxes.

\item[Output Menu]  This menu allows output of the current image(s) or
data to one of several formats.

\item[Close Button]  This button destroys the Data View Window.

\item[Set Range Button]  This button allows the user to set the color
mapping range of the data.

\item [Isometric View Buttons]  These control the view point and scaling
of the isometric image in 3D (boxes or volumetric rendering).

\begin{description}
    \item [0]              Reset the viewing angle to the original view.
    \item [XYZ]            Draw axis labels.
    \item [Draw]           Draw the volume rendered image.  This may entail
		           creating internal data structures and could take
		           several seconds to minutes for the first frame
			   to draw.
    \item [Autodraw]       Automatically render the image while rotating.
			   As with {\bf Draw}, the first image takes longer.
    \item [Trans]          Reread the transfer functions.
    \item [Detach]         Detach the isometric view into its own window.
    \item [Attach]         Attach the isometric view.
\end{description}

\end{description}


\subsubsection{Mouse Controls}

\begin{description}
\item [Left Button]  Selecting from menus, clicking buttons and animation controls,
drawing subregions in a view for datasets and subvolumes, spinning the image
in the isometric view, and clicking a point for its data value.

\item [Middle Button]  Place horizontal planar cutting lines and rotating
around the eye point in the isometric view.

\item [Right Button]   Place vertical planar cutting lines and
zooming the isometric view (pull or push the mouse).

\end{description}


\subsubsection{Batch Functions}

\begin{description}
\item [Slices] The user can specify multiple slices and ranges on the command line.
For example:  {\em amrvis -mf -xslice 7 -xslice 42 -yslice 19 -zslice 9 plt1010}
This will create four fab files of the form
{\em pltnnnn.derived.slicename.slicenum.fab}
where derived is {\em initialderived} from {\bf amrvis.defaults}.
In this example, the
first file will be (assuming pressure) {\bf plt1010.pressure.xslice.7.fab}.
The user can also specify ranges with the syntax ``start-end''  For example:
{\em -yslice 7-42} will create a fab file for each slice in the range [7,42].
Slices are specified on the finest level problem domain.
The option {\em -sliceallvars} can be specified to write all fab variables
instead of just the {\em initialderived}.
The user can specify multiple plot files as usual:  plt05*
If a {\bf zslice} is specified in 2D, {\em amrvis} will use
zslice = 0 (regardless of the
slice number specified), which is the entire dataset for the current
derived quantity.  An {\bf xslice} or {\bf yslice} in 2D creates
a strip one cell wide.
Slices are supported for any type of data {\em amrvis} can read
including fabs, which includes fabs created by slicing.  This allows
extraction of planes, lines, and points from a dataset.
Fab slices have the same {\bf BL\_SPACEDIM} as the original dataset.  A slice from
a 3D dataset is a 3D fab one cell thick.
Note:  if a slice of a dataset is created then viewed with {\em amrvis -fab},
the picture may not look the same as the original
because the dynamic range of the fab might be smaller than that of
the entire data set.  The {\bf Set Range} button can be used to set
the same values in both images to get the same picture.
This feature operates in batch mode.
{\em amrvis} will write out all requested slices for all files then exit.

\item [Making Volumetric Data Files]  Use the {\bf -makeswf} option to create
a data structure for fast volume rendering.  This option uses the
current transfer function and writes the data for processing with
a 3D movie creation application.

\end{description}



\subsection{Detailed Usage Notes}


\begin{description}
\item [Subregions] Subregions work functionally like the full region,
and the main region can be closed while retaining the subregion.
Subregions are faster for large and/or deep (many {\bf AMR} levels) files.
For deep files, the maximum level displayed is based on {\em maxpixmapsize}
and a smaller subregion must be selected to view finer levels.
A region must be selected before the {\bf Subregion} button is
pressed.

\item [Datasets]  The dataset will update its values when
the derived quantity changes.  The user can also select another region
and click on dataset (with a dataset window already showing) and it
will update that window.
A region must be selected before the {\bf Dataset} button is
pressed.

\item [File Type]
Once a file type is specified (cgmf, etc.), it cannot be changed for that
{\em amrvis} process.

\item [Animations] During animations, pressing the {\bf O} button
in the center of the animation controls will stop frame creation.
Animation is supported across files in 2D only.
It works with all supported formats and with subregions.
{\em amrvis} will automatically calculate the data range over
the file set or can be user specified with the {\bf Set Range} button.
In {\bf Set Range} for animations, the {\em File} option sets
the minimum and maximum to the values for the current data file
instead of the values over the entire set of data files.
The arrows move single frames for single
clicks and animate for double clicks.

\item [Palettes]
Palettes are in Non-interleaved Binary format and can be created
with tools such as {\em icol} from {\bf AHPCRC} or {\em ximage} from {\bf NCSA}.

\item [Number Format] The format for numbers (which is set in the
{\bf Dataset} window) affects numbers in the dataset, the set range
dialog box, and the palette.  The format value can be any valid {\bf printf}
string.  Hit the {\bf return} key after editing the string to apply.

\item [Partial colormap support]   The {\em reservesystemcolors} keyword
in {\bf amrvis.defaults} lets the user decide how many colors
to reserve from the palette.   Experiment with the terminal
to determine a satisfying number, try 20 to start.  This works
by chopping off the lower parts of the palette and remapping the
image to the new colormap.

\end{description}



\subsection{Files and Directories}

{\em Amrvis} looks for the files {\bf amrvis.defaults} and
{\bf vpramps.dat} (3D only) to
set default values and transfer functions for volume rendering.
If these files do not exist, {\em amrvis} will use standard default values.
The {\bf amrvis.defaults} file has the following structure and options
(in general: {\em keyword   value}).


\begin{verbatim}
palette               Palette
initialderived        density
initialscale          4
numberformat          %7.5f
maxpixmapsize         1000000
reservesystemcolors   20
showboxes             TRUE
windowheight          650
windowwidth           900
fabordering           sgi
fabordering           alpha
filetype              cgmf
filetype              sf
\end{verbatim}


Invalid values are ignored.  The {\em initialderived} specifies the
quantity that is displayed first and must be spelled
exactly as it appears in the data file or the value will default to
the first derived quantity.  {\em numberformat} affects the dataset and
palette values and can be any valid {\bf printf} string.
{\em windowheight} and {\em windowwidth} define the size
of the {\bf Data View Window} with arguments in number of pixels.
{\em showboxes} affects initial drawing of {\bf AMR} grid boxes.  Valid
values are {\bf TRUE} or {\bf FALSE}.
{\em fabordering} specifies the byte ordering (big or little endian)
of binary data and only affects reading old (pre {\bf BoxLib 2.0}) data files.
Valid values are {\bf sgi} for big endian and {\bf alpha} for little endian.
{\em filetype} specifies the type of file (same values as the command line).
For duplicate keywords, the last one in the file is used (for example, {\bf sf}
will be the {\em filetype} in the above file).

The search path for {\bf amrvis.defaults} is (in this order):

\begin{verbatim}
  ./amrvis.defaults
  ~/amrvis.defaults
  ~/.amrvis.defaults
\end{verbatim}


Transfer functions are set in {\bf vpramps.dat}, which has the following structure
(annotation is added here).

\begin{verbatim}
1                    (classify fields:  1==use value, 2==use value and gradient)
2                    (shade fields:     use 2 for now)
5                    (number of points in the following two lines)
0   24  180 210 255  (value field x:  first and last must be 0 and 255 for now)
0.0 0.0 0.6 0.8 1.0  (value field y:  0.0 == transparent, 1.0 == opaque)
4                    (number of points in the following two lines)
0   64  128 255      (gradient field x)
0.0 0.6 1.0 1.0      (gradient field y)
0.05                 (minRayOpacity)
0.95                 (maxRayOpacity)
\end{verbatim}



\subsection{Examples}

Typical usage is as follows.  Be sure to use the correct file type flag.

\begin{verbatim}
amrvis <return>           opens the main window with default file type (-cgmf).
amrvis -cgsf <return>     opens the main window with file type set to -cgsf.
amrvis -mf plt1560        opens plt1560 (which is a multi-fluid plot file).
amrvis -fab *.fab         opens multiple fab files in separate windows.
amrvis -sf -a plt*        opens an animation for plt*.
\end{verbatim}


\subsection{Efficiency Notes}

\begin{itemize}
\item For best performance of {\em amrvis} (and in general when
accesssing files), copy data files to local disk instead
of relying on a slow network connection.

\item In 3D, select a subregion to volume render so
{\em amrvis} can render fewer points.

\end{itemize}


\subsection{FAQ}

\begin{itemize}

\item For data sets and subregions, a region must be selected (with
the left mouse button) before clicking on {\bf Dataset} or {\bf Subregion.}
    
\item If {\em amrvis} hangs trying to read a plot file, make sure the file type
flag is set for your plot files.

\item Once you pick a file type (cgmf, etc.), you cannot change it for that
{\em amrvis} process.

\item If {\em amrvis} core dumps trying to read an old plot file, make
sure the correct fab ordering is specified in {\bf amrvis.defaults}.


\item {\em amrvis} takes background and lettering colors from
the {\bf .Xdefaults} file.
To modify these, add the following lines to {\bf .Xdefaults}

\begin{verbatim}
#ifdef COLOR
amrvis*Background:    black
amrvis*Foreground:    white
#endif
\end{verbatim}

then type

\begin{verbatim}
xrdb ~/.Xdefaults
\end{verbatim}

to update the X database.  This will turn the {\em amrvis} background to black
and the lettering white (or whatever colors are chosen).

\end{itemize}


\subsection{Known Problems}

\begin{itemize}
\item The numbers in the Palette can go off the right edge of the window
for long formats.  Also, the numbers in the printed palette are drawn as
bitmaps and not postscript text.  This causes the appearance to be jagged
in scaled printouts.

\item There is a resource allocation issue to be aware of--
{\bf amrvis.defaults} defines {\em initialscale} and {\em maxpixmapsize}.
{\em maxpixmapsize} defines the maximum number of bytes allowed to allocate
on the xterm for a pixmap (where the color picture lives) and {\em amrvis}
uses this to decide the finest level it can display.  It finds displayLevel
by:  \( (probDomain(displayLevel).numPts() * {scale^{2}}) \leq maxpixmapsize \)
The problem is this:  if the pixmap size is below but close to the
available server memory, the pixmap may be allocated properly but other
{\bf X} allocations, such as server memory for a button or menu, may fail.
This causes {\em amrvis} to die unpredictably with a core dump or an {\bf X}
message about unavailable resources.  So if this behavior happens, adjust
{\em maxpixmapsize} and {\em initialscale} values to about half
of the xterm's available memory.

\item The isometric rotation functions need to be fixed.  They are actually
correct, but for a non-intuitive frame of reference.

\item Animations for plot files with different finest levels of refinement
do not work properly.
  
\item If you specify the wrong plot file type, the behavior is undefined.
Usually, {\em amrvis} will do one of three things:  open the file with
bad numbers (Reals interpreted with the wrong byte ordering), crash, or
hang by trying to read the file forever.
  
\item When the file range for animations is set and the user derives
a different quantity, the global range value contains the value for
the current file and not the set of files in the animation.
  
\item An error with a single plot file will cause the whole application
to crash.
  
\item There are several places where redundant exposure events cause poor
performance.
  
\item The selection box does not persist during animations.

\item Cartesian grid smoothing does not always work properly in 3D.

\item The colormap does not always install properly when using twm.

\end{itemize}
